\section{findnumbersofd(n)_divisors.cpp}
\begin{lstlisting}[language=c++]
const int N = 2002;
int f[N],g[N*N];
bool ip[N];
int mu[N],p[N],nt[N];
void init_mu(){
    mu[1]=1;ip[2]=true;p[0]=2;
    for(int i=1;i<N;i+=2)   ip[i]=true;
    for(int i = 3,cnt = 1;i<N;i+=2){
        if(ip[i]){
            p[cnt++] = i;
            mu[i] = -1;
        }
        for(int j = 1,t;j<cnt&&(t= i * p[j])<N;++j){
            ip[t] = false;
            if(i % p[j] == 0)   break;
            mu[t] = -mu[i];
        }
    }
    for(int i=2;i<N;i+=4)  mu[i]=-mu[i>>1];
}
int getsum(int n){
    int sum = 0;
    for(int i=1,j;i<=n;i=j+1){
        j = n/(n/i);
        sum += (j-i+1)*(n/i);
    }
    return sum;
}
void init(){
    for(int i=0;i<N;++i)       f[i]=getsum(i);    init_mu();
    int last = 1;
    for(int i=2;i<N;++i){
        if(mu[i]==0)    continue;
        nt[last]=i;
        last = i;
    }
    nt[last]=N;
}
void getg(int a,int b,int c){
    int n = a*b;
    for(int i=0;i<=n;++i)   g[i]=0;
    for(int i=1;i<=n&&i<=c;i=nt[i]){ // n^2 \log n
        for(int j=i,t=mu[i]*f[c/i];j<=n;j+=i)   g[j]+=t;
    }
}
LL getf(int a,int b,int d){
    LL res = 0;
    for(int i=1;i<=a;++i){
        for(int j=1;j<=b;++j){
            res+=(a/i)*(b/j)*g[i*j*d];
        }
    }
    return res;
}
int main(){
//        freopen("/Users/dna049/Desktop/AC/in","r",stdin);
    //    freopen("/Users/dna049/Desktop/AC/out","w",stdout);
    init();
    int a,b,c;
    while(~scanf("%d%d%d",&a,&b,&c)){
        getg(a,b,c);
        int ab = min(a,b);
        LL res = 0;
        for(int i=1;i<=ab;i=nt[i]){
            res+=mu[i]*getf(a/i,b/i,i*i);
        }
        printf("%d\n",int(res%(1<<30)));
    }
    return 0;
}

\end{lstlisting}
\section{ISap.cpp}
\begin{lstlisting}[language=c++]
class ISap {
	/**
	 * Time Complexity: O(N^2*M)
	 * Special Case Time Complexity:
	 *		If the capacity and flow are integer, the time complexity is
	 *		O(min{N^(2/3), M^(1/2)}*M)
	 * Improve:
	 *		Gap, Multiple Augmented, Manual Stack, Active Label
	 * Usage:
	 *		init()
	 *		{addEdge()}
	 *		[preLabel()]
	 *		maxflow()
     *	Attention:
     *	    AddEdge() will add directed edge defaultly.
     *	    If you need undirected edge, please modify it.
	 *	preLabel function is not nessary. But will slower 10% if not use.
     *	In hierarchical graph, preLabel function is nessary.
	 * */
private :
	static const int N = 1001010, M = 3000010, INF = 1000000001;
public :
	int fir[N], nex[M * 2], son[M * 2], cap[M * 2], flow[M * 2], totEdge;
	int gap[N], dep[N], src, des, totPoint;

	void init(int _totPoint) {
		totEdge = 0, totPoint = _totPoint;
		for(int i = 0; i <= totPoint; ++i)
			gap[i] = dep[i] = 0, fir[i] = -1;
	}

	void addEdge(const int u, const int v, const int c) {
        /**
         * Attention, This function is added directed edge defaultly.
         * */
		nex[totEdge] = fir[u], son[totEdge] = v, cap[totEdge] = c, flow[totEdge] = 0;
		fir[u] = totEdge++;
		nex[totEdge] = fir[v], son[totEdge] = u, cap[totEdge] = c, flow[totEdge] = 0;
		fir[v] = totEdge++;
	}

	void preLabel() {
		static int que[N], head, tail;
		head = tail = 0;
		for(int i = 0; i <= totPoint; ++i) dep[i] = totPoint, gap[i] = 0;

		que[tail++] = des, dep[des] = 0, ++gap[0];
		while(head < tail) {
			int u = que[head++];
			for(int tab = fir[u], v; tab != -1; tab = nex[tab]) {
				if(flow[tab ^ 1] >= cap[tab ^ 1]) continue;
				if(dep[v = son[tab]] < totPoint) continue;
				++gap[dep[v] = dep[u] + 1], que[tail++] = v;
			}
		}
	}

	int maxflow() {
		static int cur[N], rev[N];
		int flows = 0;
		for(int i = 0; i <= totPoint; ++i) cur[i] = fir[i];
		for(int u = src; dep[u] < totPoint && dep[src] < totPoint; ) {
			if(u == des) {
				int nowFlow = INF;
				for(int p = src; p != des; p = son[cur[p]])
					nowFlow = min(nowFlow, cap[cur[p]] - flow[cur[p]]);
				for(int p = src; p != des; p = son[cur[p]])
					flow[cur[p]] += nowFlow, flow[cur[p] ^ 1] -= nowFlow;
				flows += nowFlow, u = src;
			}

			int tab;
			for(tab = cur[u]; tab != -1; tab = nex[tab])
				if(cap[tab] > flow[tab] && dep[u] == dep[son[tab]] + 1) break;
			if(tab != -1) {
				cur[u] = tab, rev[son[tab]] = tab ^ 1;
				u = son[tab];
			} else {
				if(--gap[dep[u]] == 0) break;
				cur[u] = fir[u];
				int minDep = totPoint;
				for(tab = fir[u]; tab != -1; tab = nex[tab])
					if(cap[tab] > flow[tab]) minDep = min(minDep, dep[son[tab]]);
				++gap[dep[u] = minDep + 1];
				if(u != src) u = son[rev[u]];
			}
		}
		return flows;
	}
};


\end{lstlisting}
\section{bignumber.cpp}
\begin{lstlisting}[language=c++]
class BigNum
{
	/*
	you can use BigNum(x) + BigNum(y) to calculate.
	This code doesn't use referrence when working.
	*/
	static const int MAXLENGTH = 10010, M = 100;
	// M -> MOD
public :
	int arr[MAXLENGTH], length;
	// the maximum value -> M * M * MAXLENGTH * MAXLENGTH when calculate *
	
	BigNum()
	{
		length = 0;
		clr(arr, 0);
	}
	
	BigNum(string &str) {
		length = 0, clr(arr, 0);
		int x = 0, now = 1, len = sz(str);
		for(int i = len - 1; i >= 0; i--) {
			x = x + (str[i] - '0') * now;
			now *= 10;
			if(now >= M) {
				arr[++length] = x;
				x = 0, now = 1;
			}
		}
		if(x) arr[++length] = x;
		length = max(length, 1);
	}
	
	inline void operator =(int x)
	{
		length = 0, clr(arr, 0);
		while(x)
		{
			arr[++length] = x % M;
			x /= M;
		}
		if(!length) length++;
	}
	
	BigNum(int x)
	{
		*this = x;
	}
	
	inline BigNum operator -(const BigNum& a)
	{
		BigNum ret;
		for(int i = 1; i <= length; i++) ret.arr[i] = arr[i] - a.arr[i];
		for(int i = 1; i <= length; i++)
			if(ret.arr[i] < 0)
			{
				ret.arr[i + 1]--;
				ret.arr[i] += M;
			}
		ret.length = max(length, a.length);
		while(length > 1 && !ret.arr[ret.length]) ret.length--;
		return ret;
	}
	
	inline BigNum operator +(const BigNum& a)
	{
		BigNum ret;
		ret.length = max(length, a.length);
		for(int i = 1; i <= ret.length; i++) ret.arr[i] = arr[i] + a.arr[i];
		for(int i = 1; i <= ret.length; i++)
			if(ret.arr[i] >= M)
			{
				ret.arr[i + 1]++;
				ret.arr[i] -= M;
			}
		while(ret.arr[ret.length + 1]) ret.length++;
		return ret;
	}
	
	inline BigNum operator *(const BigNum& a)
	{
		BigNum ret;
		for(int i = 1; i <= length; i++)
			for(int j = 1; j <= a.length; j++)
				ret.arr[i + j - 1] += arr[i] * a.arr[j];
		ret.length = length + a.length - 1;
		for(int i = 1; i <= ret.length; i++)
		{
			ret.arr[i + 1] += ret.arr[i] / M;
			ret.arr[i] %= M;
		}
		while(ret.arr[ret.length + 1])
		{
			ret.length++;
			ret.arr[ret.length + 1] += ret.arr[ret.length] / M;
			ret.arr[ret.length] %= M;
		}
		return ret;
	}
	
	inline BigNum operator /(int x)
	{
		BigNum ret = *this;
		for(int i = ret.length; i >= 1; i--)
		{
			if(i > 1) ret.arr[i - 1] += (ret.arr[i] % x) * M;
			ret.arr[i] /= x;
		}
		while(ret.length > 1 && !ret.arr[ret.length])
			ret.length--;
		return ret;
	}
	
	inline bool operator >(BigNum &a)
	{
		if(length != a.length) return length > a.length;
		for(int i = length; i >= 1; i--)
			if(arr[i] != a.arr[i]) return arr[i] > a.arr[i];
		return false;
	}
	
	inline void Out()
	{
		printf("%d", (int) arr[length]);
		for(int i = length - 1; i >= 1; i--)
			printf("%02d", (int) arr[i]);
		printf("\n");
	}
} ;

\end{lstlisting}
\section{MinimumArborescence_ZhuLiu.cpp}
\begin{lstlisting}[language=c++]
#define foreach(x, y) for(__typeof((y).begin()) x = (y).begin(); x != (y).end(); ++x)

// points up to 2 * n, edges will up to n * m
const int N = 1010, M = 5000010;
int st[M], ed[M], val[M];
vector<int> valid_edges, valid_points, tpoints, tedges;
int n, m, origin_n, origin_m;
int pre[N], path[N], len, belong[N], deactivate[M], activate[M];
int visit[N];
bool mark[M];

void insert(int u, int v, int w, int deact = -1, int act = -1) {
	st[m] = u, ed[m] = v, val[m] = w;
	deactivate[m] = deact, activate[m] = act;
	++m;
}

void solve() {
	origin_n = n, origin_m = m;

	for(int i = 0; i < m; ++i) valid_edges.push_back(i);
	for(int i = 0; i < n; ++i) valid_points.push_back(i);
	int answer = 0;
	int root = 0;
    bool impossible = false;
	while(true) {
		foreach(point, valid_points) pre[*point] = -1;
		foreach(e, valid_edges) {
			int v = ed[*e];
			if(pre[v] == -1 || val[pre[v]] > val[*e]) pre[v] = *e;
		}
		pre[root] = -1;

		foreach(point, valid_points)
			visit[*point] = -1, belong[*point] = *point;
		tpoints.clear(), tedges.clear();
		bool flag = false;
		foreach(point, valid_points) {
			int u = *point;
			if(visit[u] != -1) continue;
			len = 0;
			for(len = 0; visit[u] == -1; u = st[pre[u]]) {
				path[len++] = u, visit[u] = *point;
				if(pre[u] == -1) break;
			}
			if(pre[u] == -1 || visit[u] != *point) {
				for(int i = 0; i < len; ++i) tpoints.push_back(path[i]);
			} else {
                int start = 0;
                while(path[start] != u) tpoints.push_back(path[start++]);
				flag = true;
				int p = n++;
				for(int i = start; i < len; ++i) {
					int u = path[i];
					belong[u] = p;
                    mark[pre[u]] = true, answer += val[pre[u]];
				}
				tpoints.push_back(p);
			}
		}
		if(!flag) {
            int cnt_nopre = 0;
			foreach(point, valid_points)
				if(pre[*point] == -1) ++cnt_nopre;
                else answer += val[pre[*point]], mark[pre[*point]] = true;
            if(cnt_nopre > 1) impossible = true;
			break;
		}
		foreach(edge, valid_edges) {
			int u = st[*edge], v = ed[*edge], w = val[*edge];
			if(belong[u] == belong[v]) continue;
			if(belong[u] == u && belong[v] == v) tedges.push_back(*edge);
			else {
				insert(
						belong[u], belong[v], 
						belong[v] == v ? w : w - val[pre[v]], 
						belong[v] == v ? -1 : pre[v],
						*edge
						);
				tedges.push_back(m - 1);
			}
		}

		root = belong[root];
		valid_points = tpoints, valid_edges = tedges;
	}

	for(int i = m - 1; i >= 0; --i)
		if(mark[i]) {
			if(deactivate[i] != -1) mark[deactivate[i]] = false;
			if(activate[i] != -1) mark[activate[i]] = true;
		}

    if(impossible) puts("-1");
    else {
        vector<int> ans;
        for(int i = 0; i < origin_m; ++i)
            if(mark[i]) ans.push_back(i + 1);
        sort(ans.begin(), ans.end());
        assert(origin_n - 1 == (int) ans.size());
        printf("%d\n", (int) ans.size());
        for(int i = 0, s = (int) ans.size(); i < s; ++i) {
            printf(i == s - 1 ? "%d\n" : "%d ", ans[i]);
        }
        // printf("%d\n", answer);
    }
}

\end{lstlisting}
\section{zkwcostflow.cpp}
\begin{lstlisting}[language=c++]
#include <cstdio>
#include <cstring>
using namespace std;
const int maxint=~0U>>1;

int n,m,pi1,cost=0;
bool v[550];
struct etype
{
    int t,c,u;
    etype *next,*pair;
    etype(){}
    etype(int t_,int c_,int u_,etype* next_):
        t(t_),c(c_),u(u_),next(next_){}
    void* operator new(unsigned,void* p){return p;}
} *e[550];

int aug(int no,int m)
{
    if(no==n)return cost+=pi1*m,m;
    v[no]=true;
    int l=m;
    for(etype *i=e[no];i;i=i->next)
        if(i->u && !i->c && !v[i->t])
        {
            int d=aug(i->t,l<i->u?l:i->u);
            i->u-=d,i->pair->u+=d,l-=d;
            if(!l)return m;
        }
    return m-l;
}

bool modlabel()
{
    int d=maxint;
    for(int i=1;i<=n;++i)if(v[i])
        for(etype *j=e[i];j;j=j->next)
            if(j->u && !v[j->t] && j->c<d)d=j->c;
    if(d==maxint)return false;
    for(int i=1;i<=n;++i)if(v[i])
        for(etype *j=e[i];j;j=j->next)
            j->c-=d,j->pair->c+=d;
    pi1 += d;
    return true;
}

int main()
{
    freopen("costflow.in","r",stdin);
    freopen("costflow.out","w",stdout);
    scanf("%d %d",&n,&m);
    etype *Pe=new etype[m+m];
    while(m--)
    {
        int s,t,c,u;
        scanf("%d%d%d%d",&s,&t,&u,&c);
        e[s]=new(Pe++)etype(t, c,u,e[s]);
        e[t]=new(Pe++)etype(s,-c,0,e[t]);
        e[s]->pair=e[t];
        e[t]->pair=e[s];
    }
    do do memset(v,0,sizeof(v));
    while(aug(1,maxint));
    while(modlabel());
    printf("%d\n",cost);
    return 0;
}

\end{lstlisting}
\section{makrpointbypolygon.cpp}
\begin{lstlisting}[language=c++]
const int N = 610, M = 8010 * 2;
const double EPS = 1e-7;
struct Point {
	int x, y;

	Point() {}
	Point(int x, int y):x(x), y(y) {}

	inline Point operator +(const Point &t) const {
		return Point(x + t.x, y + t.y);
	}

	inline Point operator -(const Point &t) const {
		return Point(x - t.x, y - t.y);
	}

	inline ll operator *(const Point &t) const {
		return x * t.x + y * t.y;
	}

	inline ll operator %(const Point &t) const {
		return x * t.y - y * t.x;
	}

	inline bool operator <(const Point &t) const {
		return x < t.x || (x == t.x && y < t.y);
	}

	inline void read() {
		scanf("%d%d", &x, &y);
	}
} cap[N], p[M * 2];
double arg[M * 2];
typedef vector<int> Poly;
int n, m;
map<Point, vector<int> > graph;
map<int, int> ofPoly;
int father[N], bel[N];
vector<int> cover[N];
Poly poly[M];
int totPoly, con[M];
double area[M];
vector<int> ans[N];
bool visit[M];
int idx[N];

inline void init() {
	graph.clear(), ofPoly.clear();
	for(int i = 0; i < n; ++i) cover[i].clear();
	for(int i = 0; i < n; ++i) ans[i].clear();
	clr(visit, 0), totPoly = 0;
}

inline int sgn(double x) {
	return x < -EPS ? -1 : x > EPS;
}

inline double getArg(const Point &a) {
	return atan2(a.y, a.x);
}

inline bool cmpByArg(int a, int b) {
	return arg[a] < arg[b];
}

inline bool cmpByArea(int a, int b) {
	return area[a] < area[b];
}

inline bool cmpByOfPolyArea(int x, int y) {
	return area[bel[x]] < area[bel[y]];
}

inline int getNextEdge(int x) {
	int rank = lower_bound(all(graph[p[x ^ 1]]), x ^ 1, cmpByArg) - 
				graph[p[x ^ 1]].begin();
	return rank ? graph[p[x ^ 1]][rank - 1] : graph[p[x ^ 1]].back(); 
}

inline double getArea(const Poly &n) {
	double ret = 0.;
	for(int i = 1; i < sz(n) - 1; ++i)
		ret += (p[n[i]] - p[n[0]]) % (p[n[i + 1]] - p[n[0]]);
	return ret / 2.;
}

inline bool pointInPoly(const Point &x, const Poly &n) {
	int cross = 0;
	for(int i = 0; i < sz(n); ++i) {
		Point v1 = p[n[i]] - x, v2 = p[n[(i + 1) % sz(n)]] - x;
		int c = sgn(v1 % v2), d = sgn(v1 * v2);
		if(!c && d <= 0) return 0;
		int d1 = sgn(v1.y), d2 = sgn(v2.y);
		if(c > 0 && d1 <= 0 && d2 > 0) ++cross;
		if(c < 0 && d2 <= 0 && d1 > 0) --cross;
	}
	return cross > 0;
}

inline void solve() {
	// build graph by angle
	for(int i = 0; i < 2 * m; ++i) arg[i] = getArg(p[i ^ 1] - p[i]);
	for(int i = 0; i < 2 * m; ++i) graph[p[i]].pub(i);
	foreach(it, graph) sort(all(it->second), cmpByArg);

	// get all polygons
	for(int i = 0; i < 2 * m; ++i) {
		if(visit[i]) continue;
		Poly t;
		t.clear();
		int now = i;
		do {
			visit[now] = true, t.pub(now);
			now = getNextEdge(now); 
		} while(now != i);
		area[totPoly] = getArea(t);
		if(sgn(area[totPoly]) <= 0) continue;
		poly[totPoly] = t;
		foreach(it, t) ofPoly[*it] = totPoly;
		for(int j = 0; j < n; ++j)
			if(pointInPoly(cap[j], t)) cover[j].pub(totPoly);
		++totPoly;
	}

	for(int i = 0; i < n; ++i) sort(all(cover[i]), cmpByArea);
	for(int i = 0; i < n; ++i) con[bel[i] = cover[i][0]] = i;
	for(int i = 0; i < 2 * m; i += 2)
		if(ofPoly.count(i) && ofPoly.count(i ^ 1)) {
			int x = con[ofPoly[i]], y = con[ofPoly[i ^ 1]];
			// printf("%d %d\n", x, y);
			ans[x].pub(y), ans[y].pub(x);
		}

	for(int i = 0; i < n; ++i) idx[i] = i;
	sort(idx, idx + n, cmpByOfPolyArea);
	for(int i = 0; i < n; ++i) {
		bool flag = true;
		foreach(e, poly[bel[idx[i]]])
			if(ofPoly.count(*e ^ 1) == 0) {
				flag = false;
				break;
			}
		if(!flag) {
			for(int j = i + 1; j < n; ++j)
				if(pointInPoly(cap[idx[i]], poly[bel[idx[j]]])) {
					ans[idx[i]].pub(idx[j]), ans[idx[j]].pub(idx[i]);
					break;
				}
		}
	}

	for(int i = 0; i < n; ++i) {
		sort(all(ans[i]));
		ans[i].erase(unique(all(ans[i])), ans[i].end());
		printf("%d", sz(ans[i]));
		for(int j = 0; j < sz(ans[i]); ++j)
			printf(" %d", ans[i][j] + 1);
		puts("");
	}
}

int main() {
	freopen("a.in", "r", stdin);
	freopen("a.out", "w", stdout);
	while(scanf("%d%d", &n, &m) == 2 && n + m != 0) {
		init();
		for(int i = 0; i < n; ++i) cap[i].read();
		for(int i = 0; i < 2 * m; ++i) p[i].read();
		solve();
	}
	return 0;
}


\end{lstlisting}
\section{fft.cpp}
\begin{lstlisting}[language=c++]
class Complex {
	public :
		double real, image;

		Complex(double real = 0., double image = 0.):real(real), image(image) {}
		Complex(const Complex &t):real(t.real), image(t.image) {}

		Complex operator +(const Complex &t) const {
			return Complex(real + t.real, image + t.image);
		}

		Complex operator -(const Complex &t) const {
			return Complex(real - t.real, image - t.image);
		}

		Complex operator *(const Complex &t) const {
			return Complex(real * t.real - image * t.image, 
						real * t.image + t.real * image);
		}
};

const int N = 300010, MOD = 313;
const double PI = acos(-1.);

class FFT {
	/**
	 * 1. Need define PI
	 * 2. Need define class Complex
	 * 3. tmp is need for fft, so define a N suffice it
	 * 4. dig[30] -> (1 << 30) must bigger than N
	 * */
	private :
		static Complex tmp[N];
		static int revNum[N], dig[30];

	public :
		static void init(int n) {
			int len = 0;
			for(int t = n - 1; t; t >>= 1) ++len;
			for(int i = 0; i < n; i++) {
				revNum[i] = 0;
				for(int j = 0; j < len; j++) dig[j] = 0;
				for(int idx = 0, t = i; t; t >>= 1) dig[idx++] = t & 1;
				for(int j = 0; j < len; j++)
					revNum[i] = (revNum[i] << 1) | dig[j];
			}
		}

		static int rev(int x) {
			return revNum[x];
		}

		static void fft(Complex a[], int n, int flag) {
			/**
			 * flag = 1 -> DFT
			 * flag = -1 -> IDFT
			 * */
			for(int i = 0; i < n; ++i) tmp[i] = a[rev(i)];
			for(int i = 0; i < n; ++i) a[i] = tmp[i];
			for(int i = 2; i <= n; i <<= 1) {
				Complex wn(cos(2 * PI / i), flag * sin(2 * PI / i));
			   for(int k = 0, half = i / 2; k < n; k += i) {
				   Complex w(1., 0.);
				   for(int j = k; j < k + half; ++j) {
					   Complex x = a[j], y = w * a[j + half];
					   a[j] = x + y, a[j + half] = x - y;
					   w = w * wn;
				   }
			   }
			}
			if(flag == -1) {
				for(int i = 0; i < n; ++i) a[i].real /= n;
			}
		}

		static void dft(Complex a[], int n) {
			fft(a, n, 1);
		}

		static void idft(Complex a[], int n) {
			fft(a, n, -1);
		}
};
Complex FFT::tmp[N];
int FFT::revNum[N], FFT::dig[30];


\end{lstlisting}
\section{FastIO.cpp}
\begin{lstlisting}[language=c++]
namespace FastIO {
    /**
     * Defaultly, can deal with negative number.
     * Only for reading integers, modify for specify problems.
     * Throws int exception if nothing to read.
     * */
    const int S = 2000000;
    char s[S], *h = s+S, *t = h;
    inline char getchr(void) {
        if(h == t) {
            if(t != s + S) return -1;
            t = s + fread(s, 1, S, stdin), h = s;
        }
        return *h++;
    }
    inline int getint(void) throw(int) {
        char c = ' ';
        bool positive = true;
        for (; !isdigit(c); c = getchr()) {
            if(c == -1) throw(-1);
            positive ^= (c == '-');
        }
        int x = 0;
        for (; isdigit(c); c = getchr()) x = x * 10 + c - '0';
        return positive ? x : -x;
    }
}
using FastIO::getint;

\end{lstlisting}
\section{sustainable01tree.cpp}
\begin{lstlisting}[language=c++]
struct Trie {
	/**
	 * 1. MAXBIT = max depth
	 * 2. rot = The root of the tr[i](ith Trie)
	 * 3. interface add(val, w) -> val (value)   w (times)
	 *		insert w elements with value val to the trie
	 * 4. remember initialize the Trie::Node::tot = 0
	 * 5. updata -  change it as you want, it will updata each
	 *	node after insert value to its children. In the template,
	 *	it updata f 
	 * */
	static struct Node {
		static int tot;

		int child[2], size;

		int f;

		inline void init() {
			child[0] = child[1] = -1, size = f = 0;
		}

	} tr[N * M * M];

#define child(x, y) tr[x].child[y]
#define lch(x) child(x, 0)
#define rch(x) child(x, 1)
#define size(x) (x == -1 ? 0 : tr[x].size)
#define f(x) tr[x].f

	int rot;

	inline int bruteForce(int u, int v, int d) {
		int ret = 0;
		while(d >= 0) {
			int t = size(lch(v)) == 0;
			int _t = t;
			if(size(child(u, _t)) == 0) _t ^= 1, ret += (1 << d);
			u = child(u, _t), v = child(v, t), --d;
		}
		return ret;
	}

	inline void updata(int u, int d) {
		f(u) = -1;
		if(size(u) > 1) {
			f(u) = 0;
			for(int t = 0; t < 2; ++t) {
				int v = child(u, t);
				if(size(v) > 1) f(u) = max(f(u), f(v));
				else if(size(v) == 1) {
					int xorValue = bruteForce(child(u, t ^ 1), v, d - 1);
					xorValue += (1 << d);
					f(u) = max(f(u), xorValue);
				}
			}
		}
	}

	inline void addVal(int &x, int val, int w, int d) {
		if(x == -1) tr[x = Node::tot++].init();
		tr[x].size += w;
		if(d >= 0) addVal(child(x, (val & (1 << d)) > 0), val, w, d - 1);

		updata(x, d);
	}

	inline void add(int val, int w) {
		addVal(rot, val, w, MAXBIT - 1);
	}

	inline void traverse(int x, int now, int d, Trie &pro) {
		if(x == -1) return;
		if(d < 0) pro.add(now, size(x));
		else {
			if(size(lch(x)) > 0)
				traverse(lch(x), now, d - 1, pro);
			if(size(rch(x)) > 0)
				traverse(rch(x), now + (1 << d), d - 1, pro);
		}
	}

	inline void operator +=(Trie &t) {
		if(size(rot) < size(t.rot)) swap(rot, t.rot);
		t.traverse(t.rot, 0, MAXBIT - 1, *this);
	}

	inline int getAnswer() const {
		return f(rot);
	}

} tr[N];
int Trie::Node::tot;
Trie::Node Trie::tr[N * M * M];


\end{lstlisting}
\section{printitself.cpp}
\begin{lstlisting}[language=c++]
#include<iostream>
#include<string>
using namespace std;
/**************************************
* Welcome to visit http://dna049.com
**************************************/
string a[20];
int main(){
a[0]="#include<iostream>";
a[1]="#include<string>";
a[2]="using namespace std;";
a[3]="/**************************************";
a[4]="* Welcome to visit http://dna049.com";
a[5]="**************************************/";
a[6]="string a[20];";
a[7]="int main(){";
a[8]="for(int i=0;i<8;++i) cout<<a[i]<<endl;";
a[9]="for(int i=0;i<=12;++i) cout<<char(97)<<char(91)<<i<<char(93)<<char(61)<<char(34)<<a[i]<<char(34)<<char(59)<<endl;";
a[10]="for(int i=8;i<=12;++i) cout<<a[i]<<endl;";
a[11]="return 0;";
a[12]="}";
for(int i=0;i<8;++i) cout<<a[i]<<endl;
for(int i=0;i<=12;++i) cout<<char(97)<<char(91)<<i<<char(93)<<char(61)<<char(34)<<a[i]<<char(34)<<char(59)<<endl;
for(int i=8;i<=12;++i) cout<<a[i]<<endl;
return 0;
}

\end{lstlisting}
\section{karpminimumcircuit.cpp}
\begin{lstlisting}[language=c++]
#include <cstdio>
#include <cstring>
#include <cstdlib>
#include <cmath>
#include <ctime>
#include <iostream>
#include <algorithm>
#include <set>
#include <map>
#include <queue>
#include <vector>
#include <deque>
using namespace std;
#define INF (1000000001)
#define MLL (1000000000000000001LL)
#define puf push_front
#define pub push_back
#define pof pop_front
#define pob pop_back
#define sz(x) ((int) (x).size())
typedef long long ll;
typedef unsigned long long ull;

template<class T>
inline T read() {
	char ch = ' ';
	T ret = (T) 0;
	bool positive = 1;
	while(!(ch >= '0' && ch <= '9')) {
		if(ch == '-') positive ^= 1;
		ch = getchar();
	}
	while(ch >= '0' && ch <= '9') {
		ret = ret * ((T) 10) + ((T) (ch - '0'));
		ch = getchar();
	}
	return positive ? ret : -ret;
}

template<class T>
inline void print(T x) {
    static int a[24];
	int n = 0;
    while(x > 0) {
        a[n++] = x % 10;
        x /= 10;
    }
    if(n == 0) a[n++] = 0;
    while(n--) putchar('0' + a[n]);
}

const int N = 110, M = 100 * 100 + 10;
int n, m;
int dis[N][N];
struct EdgeNode {
	int u, v, c;
	EdgeNode() {}
	EdgeNode(int u, int v, int c):u(u), v(v), c(c) {}

	inline void read() {
		scanf("%d%d%d", &u, &v, &c);
	}
} edges[M];

inline void solve() {
	for(int i = 0; i < m; i++) edges[i].u--, edges[i].v--;

	for(int i = 0; i < n; i++) dis[0][i] = INF;
	dis[0][0] = 0;
	for(int i = 1; i <= n; i++) {
		for(int j = 0; j < n; j++)
			dis[i][j] = INF;

		for(int j = 0; j < m; j++)
			if(dis[i][edges[j].v] > dis[i - 1][edges[j].u] + edges[j].c)
				dis[i][edges[j].v] = dis[i - 1][edges[j].u] + edges[j].c;
	}

	double ans = INF;
	for(int i = 0; i < n; i++) {
		if(dis[n][i] >= INF) continue;
		double now = -INF;
		for(int j = 0; j < n; j++)
			if(dis[j][i] < INF)
				now = max(now, (dis[n][i] - dis[j][i]) / (1. * (n - j)));
		if(now > -INF) ans = min(ans, now);
	}

	printf("%lf\n", ans);
}

int main() {
	freopen("a.in", "r", stdin);
	scanf("%d%d", &n, &m);
	for(int i = 0; i < m; i++) edges[i].read();
	solve();
	return 0;
}


\end{lstlisting}
\section{dancinglinks.cpp}
\begin{lstlisting}[language=c++]
struct DLX {
	/**
	1.	node coordinate (1...r, 1...c)
	2.	the order of adding nodes should be 
		from up to down, from left to right
	*/
	static const int ROWS = 125 * 5 + 10., COLS = 125 + 10;
	static const int N = ROWS * COLS + ROWS + COLS;
	int rows, cols;
	int u[N], d[N], l[N], r[N], rowIndex[N], colIndex[N], nodes;
	int countCols[N];
	
	inline void init(int _rows, int _cols) {
		rows = _rows, cols = _cols;
		nodes = 0;
		for(int i = 0; i <= cols; i++) {
			int now = nodes++;
			u[now] = d[now] = now;
			l[now] = i == 0 ? cols : i - 1;
			r[now] = i == cols ? 0 : i + 1;
			rowIndex[now] = 0, colIndex[now] = i;
		}
		for(int i = 1; i <= rows; i++) {
			int now = nodes++;
			u[now] = i == 1 ? rows + cols : i - 1 + cols;
			d[now] = i == rows ? 1 + cols : i + 1 + cols;
			l[now] = r[now] = now;
			rowIndex[now] = i, colIndex[now] = 0;
		}
		for(int i = 0; i <= cols; i++) countCols[i] = 0;
	}
	
	inline void addNode(int x, int y) {
		int now = nodes++;
		u[now] = u[y], d[u[y]] = now;
		d[now] = y, u[y] = now;
		l[now] = l[cols + x], r[l[cols + x]] = now;
		r[now] = cols + x, l[cols + x] = now;
		countCols[y]++;
		rowIndex[now] = x, colIndex[now] = y;
	}
	
	inline void del(int x) {
		/**
		the current del node x will be delete first or
		never show as a real node and act as a virtual node
		*/
		if(!x) return;
		r[l[x]] = r[x], l[r[x]] = l[x];
		for(int row = d[x]; row != x; row = d[row])
			for(int col = r[row]; col != row; col = r[col]) {
				u[d[col]] = u[col], d[u[col]] = d[col];
				countCols[colIndex[col]]--;
			}
	}
	
	inline void resume(int x) {
		if(!x) return;
		for(int row = u[x]; row != x; row = u[row])
			for(int col = l[row]; col != row; col = l[col]) {
				u[d[col]] = col, d[u[col]] = col;
				countCols[colIndex[col]]++;
			}
		r[l[x]] = l[r[x]] = x;
	}
	
	inline void dance(bool *ans, int *size, int used, int remain) {
		/**
		change here
		*/
		ans[used + remain] = true;
		
		int p = 0, maxx = -INF;
		for(int tab = r[0]; tab != 0; tab = r[tab])
			if(countCols[tab] > maxx)
				maxx = countCols[p = tab];
		
		if(!p) return;
		del(p);
		for(int row = d[p]; row != p; row = d[row]) {
			for(int col = r[row]; col != row; col = r[col])
				del(colIndex[col]);
			dance(ans, size, used + 1, remain - size[rowIndex[row]]);
			for(int col = l[row]; col != row; col = l[col])
				resume(colIndex[col]);
		}
		resume(p);
	}
} dlk;
\end{lstlisting}
\section{blocklist.cpp}
\begin{lstlisting}[language=c++]
struct BlockList {
	/**
	1.	indexes -> [1..n].
	2.	each block has at least one element after delete.
	3.	remenber to updata head, tail after each operation.
	*/
	const static int MAXLEN = N * 2;
	
	int l[MAXLEN], r[MAXLEN], val[MAXLEN];
	int n, block_len, blocks/*, startUse*/;
	int head[MAXLEN], tail[MAXLEN], size[MAXLEN];
	
	int space[MAXLEN];
	
	inline void init(int len) {
		/**
		O(len)
		*/
		n = len;
		for(block_len = 2; block_len * block_len <= n; block_len++) ;
		// block_len = 700;
		for(blocks = 1; blocks * block_len < n; blocks++) ;
		for(int i = 1; i <= blocks; i++) size[i] = 0;
		
		int now = 0;
		size[0] = block_len;
		for(int i = 1; i <= n; i++) {
			l[i] = i - 1, r[i] = i + 1;
			val[i] = 0;
			
			if(size[now] >= block_len) head[++now] = i;
			size[now]++, tail[now] = i;
		}
		r[n] = 0;
		l[0] = r[0] = 0;
		
		/*(int i = n + 1; i < MAXLEN - 1; i++) r[i] = i + 1;
		startUse = n + 1;*/
		
		for(int i = 1; i <= blocks; i++)
			space[i] = 1;
	}
	
	inline int getBlockIndex(int x) {
		/**
		return index of the block that xth element belongs to.
		O(sqrt(n))
		*/
		int ret = 1, nowCount = size[1];
		while(ret < blocks && nowCount < x) nowCount += size[++ret];
		return nowCount < x ? -1 : ret;
	}
	
	inline int at(int x) {
		/**
		get xth element of blocklist
		O(sqrt(n))
		*/
		int bindex = getBlockIndex(x), now = 0;
		if(bindex == -1) return -1;
		for(int i = 1; i < bindex; i++) now += size[i];
		int ret = head[bindex];
		for(++now; now < x; ++now) ret = r[ret];
		return ret;
	}
	
	inline int getFirstSpace(int st = 1) {
		int bindex = getBlockIndex(st), x = at(st);
		while(true) {
			if(val[x] == 0) return st;
			if(x == tail[bindex]) break;
			x = r[x], st++;
		}
		
		for(int i = bindex + 1; i <= blocks; i++) {
			if(space[i] != -1) return st + space[i];
			st += size[i];
		}
		return -1;
	}
	
	inline void calcSpace(int bindex) {
		space[bindex] = 1;
		int tab = head[bindex];
		while(true) {
			if(val[tab] == 0) return;
			if(tab == tail[bindex]) break;
			tab = r[tab], space[bindex]++;
		}
		space[bindex] = -1;
	}
	
	inline void set(int pos, int v) {
		/**
		set pos-th element's value as v
		O(sqrt(n))
		*/
		int x = at(pos), bindex = getBlockIndex(pos);
		val[x] = v;
		
		calcSpace(bindex);
	}
	
	inline void move(int pos, int value) {
		int x = at(pos), ypos = getFirstSpace(pos);
		int bix = getBlockIndex(pos), biy = getBlockIndex(ypos);
		int y = at(ypos);
		
		if(pos != ypos) {
			if(l[y]) r[l[y]] = r[y];
			if(r[y]) l[r[y]] = l[y];
			if(tail[biy] == y) tail[biy] = l[y];
			if(head[biy] == y) head[biy] = r[y];
			if(l[x]) r[l[x]] = y;
			l[y] = l[x];
			l[x] = y, r[y] = x;
			if(head[bix] == x) head[bix] = y;
			size[bix]++, size[biy]--;
			
			for(int i = bix + 1; i <= biy; i++) {
				if(val[tail[i - 1]] == 0) space[i] = 1;
				else if(space[i] != -1) space[i]++;
				if(space[i - 1] == size[i - 1]) space[i - 1] = -1;
				head[i] = tail[i - 1];
				tail[i - 1] = l[tail[i - 1]];
				size[i]++, size[i - 1]--;
			}
		}
		
		set(pos, value);
		calcSpace(bix), calcSpace(biy);
	}
	
	inline void getlist(int *ret, int &len) {
		len = 0;
		for(int i = 1; i <= blocks; i++) {
			int tab = head[i];
			while(true) {
				ret[++len] = val[tab];
				if(tab == tail[i]) break;
				tab = r[tab];
			}
		}
	}
} mylist;
\end{lstlisting}
\section{treehash.cpp}
\begin{lstlisting}[language=c++]
const int PRIMESTOT = 7;
const unsigned int PRIMES[PRIMESTOT] = {
	313,
	5483, 
	85017, 
	451883, 
	6459271,
	37562047,
	142859339
};

struct cmpByHashCode {
	unsigned int *keys;
	cmpByHashCode(unsigned int *keys):keys(keys) {}
	inline bool operator ()(const int a, const int b) const {
		return keys[a] < keys[b];
	}
};

struct TreeHash {
	/**
	 * 1. It will find tree's barycentre firstly and treat it as root.
	 * 2. If it has two barycentres, it will add a new nodes between them.
	 * 3. Then Hash Every nodes by it subtree's structure.
	 * 4. O(n * number of primes).
	 * 5. Choose several primes as keys to hash.
	 * 6. The number of primes determine the accuracy of hash.
	 * */

	static unsigned int factor[N];

	static void prepare(int N) {
		unsigned int tmp[PRIMESTOT];
		for(int i = 0; i < PRIMESTOT; ++i) tmp[i] = PRIMES[i];
		for(int i = 0; i < N; ++i) {
			factor[i] = tmp[i % PRIMESTOT];
			tmp[i % PRIMESTOT] *= tmp[i % PRIMESTOT];
		}
	}

	int head[N], son[N * 2], nex[N * 2], tot;
	int n;
	int rot, bfsList[N], father[N], size[N], maxSubtree[N];
	unsigned int hashCode[N];
	vector<int> child[N];

	inline void init(int m) {
		n = m;
		for(int i = 0; i < n; ++i) head[i] = -1;
		tot = 0;
	}

	inline void addEdge(int u, int v) {
		son[tot] = v, nex[tot] = head[u];
		head[u] = tot++;
	}

	inline void bfs(int st) {
		int len = 0;
		bfsList[len++] = st, father[st] = -1;
		for(int idx = 0; idx < len; ++idx) {
			int u = bfsList[idx];
			for(int tab = head[u], v; tab != -1; tab = nex[tab])
				if(father[u] != (v = son[tab]))
					father[v] = u, bfsList[len++] = v;
		}
	}

	inline int getBarycentre() {
		bfs(0);

		for(int i = 0; i < n; ++i) size[i] = 0;
		for(int i = n - 1; i >= 0; --i) {
			int u = bfsList[i];
			++size[u];
			if(father[u] != -1) size[father[u]] += size[u];

			maxSubtree[u] = n - size[u];
			for(int tab = head[u], v; tab != -1; tab = nex[tab])
				if(father[u] != (v = son[tab]))
					maxSubtree[u] = max(maxSubtree[u], size[v]);
		}

		int rot = 0;
		for(int i = 1; i < n; ++i)
			if(maxSubtree[rot] > maxSubtree[i]) rot = i;
		int anotherRot = -1;
		for(int i = 0; i < n; ++i)
			if(i != rot && maxSubtree[rot] == maxSubtree[i]) {
				anotherRot = i;
				break;
			}

		if(anotherRot != -1) {
			int newRot = n++;
			head[newRot] = -1;
			addEdge(newRot, rot), addEdge(newRot, anotherRot);

			for(int tab = head[rot]; tab != -1; tab = nex[tab])
				if(son[tab] == anotherRot) {
					son[tab] = newRot;
					break;
				}
			for(int tab = head[anotherRot]; tab != -1; tab = nex[tab])
				if(son[tab] == rot) {
					son[tab] = newRot;
					break;
				}

			rot = newRot;
		}
		return rot;
	}

	inline int hashTree() {
		int rot = getBarycentre();
		bfs(rot);

		for(int i = n - 1; i >= 0; --i) {
			int u = bfsList[i];

			child[u].clear(), hashCode[u] = 1;
			for(int tab = head[u], v; tab != -1; tab = nex[tab])
				if(father[u] != (v = son[tab])) child[u].push_back(v);
			sort(all(child[u]), cmpByHashCode(hashCode));
			for(int idx = 0; idx < sz(child[u]); ++idx)
				hashCode[u] += factor[idx] * hashCode[child[u][idx]];
		}
		return rot;
	}
};
unsigned int TreeHash::factor[N];


\end{lstlisting}
\section{countprimes.cpp}
\begin{lstlisting}[language=c++]
const int N = 6e6+2;
bool np[N];
int p[N],pi[N];
int getprime(){
    int cnt=0;
    np[0]=np[1]=true;
    pi[0]=pi[1]=0;
    for(int i = 2; i < N; ++i){
        if(!np[i]) p[++cnt] = i;
        for(int j = 1;j <= cnt && i * p[j] < N; ++j) {
            np[i * p[j]] = true;
        }
        pi[i]=cnt;
    }
    return cnt;
}
const int M = 7;
const int PM = 2*3*5*7*11*13*17;
int phi[PM+1][M+1],sz[M+1];
void init(){
    getprime();
    sz[0]=1;
    for(int i=0;i<=PM;++i)  phi[i][0]=i;
    for(int i=1;i<=M;++i){
        sz[i]=p[i]*sz[i-1];
        for(int j=1;j<=PM;++j){
            phi[j][i]=phi[j][i-1]-phi[j/p[i]][i-1];
        }
    }
}
int sqrt2(LL x){
    LL r = (LL)sqrt(x-0.1);
    while(r*r<=x)   ++r;
    return int(r-1);
}
int sqrt3(LL x){
    LL r = (LL)cbrt(x-0.1);
    while(r*r*r<=x)   ++r;
    return int(r-1);
}
LL Phi(LL x,int s){
    if(s == 0)  return x;
    if(s <= M)  return phi[x%sz[s]][s]+(x/sz[s])*phi[sz[s]][s];
    if(x <= p[s]*p[s])   return pi[x]-s+1;
    if(x <= p[s]*p[s]*p[s] && x< N){
        int s2x = pi[sqrt2(x)];
        LL ans = pi[x]-(s2x+s-2)*(s2x-s+1)/2;
        for(int i=s+1;i<=s2x;++i){
            ans += pi[x/p[i]];
        }
        return ans;
    }
    return Phi(x,s-1)-Phi(x/p[s],s-1);
}
LL Pi(LL x){
    if(x < N)   return pi[x];
    LL ans = Phi(x,pi[sqrt3(x)])+pi[sqrt3(x)]-1;
    for(int i=pi[sqrt3(x)]+1,ed=pi[sqrt2(x)];i<=ed;++i){
        ans -= Pi(x/p[i])-i+1;
    }
    return ans;
}
int main(){
    init();
    LL n;
    while(scanf("%lld",&n)!=EOF) {
        printf("%lld\n",Pi(n));
    }
    return 0;
}


\end{lstlisting}
\section{superpower.cpp}
\begin{lstlisting}[language=c++]
inline int superPower(int k, int p) {
	/**
	 * 1. k^X mod p = k^(X mod phi(p) + phi(p)) mod p  X = k^k^k.....
	 * 2. fastpow(x, y, z) x^y mod z
	 * */
	if(p == 1) return 0;
	int powers = superPower(k, phi[p]) + phi[p];
	return fastpow(k, powers, p);
}


\end{lstlisting}
